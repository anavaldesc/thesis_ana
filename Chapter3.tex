% !TEX root = mainthesis.tex

%Chapter 3


\renewcommand{\thechapter}{3}

\chapter{Coherent manipulation and detection of ultra-cold atoms}
\label{ch:Ch3}

All of the experiments described in this thesis were performed using ultracold clouds of $\Rb87$. Both the cooling and trapping of atoms as well as the engineering of potentials and detection of atoms rely on the interaction of atoms with electromagnetic fields as well as with static and oscillating magnetic fields. 

In this Chapter I will describe the techniques and interactions that make our experiments possible. I will start by describing the electronic structure of $\Rb87$. I will review the effects of the atomic interactions with magnetic and electric fields. First I will talk about the foundations of atom-light interactions that make possible both laser cooling and trapping and Raman induced transitions. I will then talk about the interactions of atoms with magnetic fields which allows us to shift the energies of different atomic states and . Finally I will discuss the absorption imaging technique that we use to detect atoms after all our experiments are performed. 

\section{Electronic structure of $^{87}$Rb}
\label{sec:electronic_structure}

Rb is an Alkali metal (also Li, which exists in our vacuum chamber but was never used). Alkali metals correspond to the first group (leftmost column) of the periodic table and are characterized by having a single valence electron, which makes the description of their internal structure much simpler than that of other elements. We can describe the state of an electron in an atom by its angular momentum $\mathbf{\hat{L}}$ and its spin $\mathbf{\hat S}$. Because of Pauli's exclusion principle there can not be two electrons with the same quantum numbers and in multi-electron atoms they tend to fill `shells' of different angular momentum values, historically labeled by the letters $S,\ P,\ D,\ F,\ ...$\footnote{This terms were used to describe the lines in the emission spectra when they were first discovered. $S$ stands for sharp, $P$ for principal $D$ for diffuse and $F$ for further noted} (corresponding to $L=1,\ 2,\ 3,\ 4,\ ...$). In particular Rb has 4 filled shells and one electron in the $5S$ 
shell, where the number $5$ corresponds to the principal quantum number $n$. Figure \note{TODO: make figure of atomic energy levels} shows the energy levels of the ground state $5S$ and its closest $5P$ orbital. %In the absence of interactions, the $m_l$ sublevels within an orbital are degenerate.

The atomic level structure is modified by relativistic effects. In particular the relativistic treatment of the electron's motion gives rise to an interaction between the electron's intrinsic magnetic moment (the spin) $\mathbf{\hat S}$ and the orbital angular momentum $\mathbf{\hat L}$. This spin-orbit coupling interaction $\hat H_{\rm {fs}} \propto \mathbf{L}\cdot\mathbf{S}$ causes the fine structure splitting of the electronic orbitals into levels with different total electronic angular momentum $\mathbf{J}=\mathbf{\hat L}\cdot\mathbf{\hat S}$. Figure~\ref{fig:fs_hfs}b show the $5^2S_{1/2}$, $5^2P_{1/2}$ and $5^2P{3/2}$ electronic configurations that arise from this splitting, where the subscript indicates the value of $J$. For $S$ ($L=0$) orbitals $J=1/2$ is the only possible value and the levels are not split. For the $P$ orbital ($L=1$) $J$ and a single electron with $S=1/2$, $J$ can be $1/2$ or $3/2$ and the $P$ orbital splits into two levels. The $5S_{1/2}\rightarrow 5P_{1/2}$ is known as the D1 line and has wavelength $\lambda=\unit[794.979]{nm}$ and $5S_{1/2}\rightarrow 5P_{3/2}$ transition is known as the D2 line and has $\lambda=\unit[790.241]{nm}$ \cite{Steck}. 

The atomic level structure gets further modified by the magnetic interaction of the electronic magnetic flux density with the nuclear spin $\mathbf{I}$. This is another kind of spin-orbit interaction that gives rise to the hyperfine splitting of the atomic levels which can be described by the Hamiltonian $\hat H_{\rm{hfs}} = A_{\rm{hfs}}\mathbf{I}\cdot\mathbf{J}$. A complete derivation of $\hat H_{\rm{hfs}}$ can be found in~\cite{schwartz_theory_1955}. The hyperfine levels correspond to different values of the total angular momentum $\hat F=\hat J+\hat I$. For $\Rb87$ $I=3/2$~\cite{Steck} which results in the level structure shown in Figure~\ref{fig:fs_hfs}c 

\section{Atom-light interaction}
\label{sec:atom-lignt_interaction}

In this section I will discuss the interaction between atoms and electromagnetic radiation (light). After laying the foundations I will discuss the applications using off-resonant light such as optical dipole traps and Raman transitions. I will not cover laser cooling which has been covered extensively in the literature~\cite{metcalf_deceleration_1999,phillips_nobel_1998} and PhD theses from previous lab members~\cite{CampbellThesis,PriceThesis}. 

In the presence of an electric field $\mathbf E$ an atom can become polarized and therefore its energy levels get modified by the Stark effect~\cite{stark_beobachtungen_1914}. If the electric field is spatially uniform with respect to the atom's size we consider the electric field as a classical object and its effect on the atom can be described by the the Hamiltonian~\cite{Cohen-Tanoudji}  
%
\begin{equation}
\hat{H}_{\rm{dip}} = -\mathbf{\hat d}\cdot\mathbf{E},
\label{eq:dipole_ham}	
\end{equation}
%
where $\mathbf{\hat d}=-e\sum_j r_j$ is the atomic dipole operator, $e$ is the electron charge and $\hat r_j$ are the position operators of the atom's electrons relative to the center of mas of the atom. This approximation, known as the dipole approximation, and is valid for electromagnetic radiation when the wavelength is much larger than the size of an atom $\lambda\gg r_{\rm{atom}}$~\cite{SteckTextbook}. 

For a coherent electromagnetic field $\mathbf{E}(\omega,t)$ with angular frequency $\omega$, the dipole Hamiltonian can be written in terms of a dynamic polarizability
%
\begin{equation}
	\hat{H}_{\rm{dip}}=-\alpha_{\mu\nu}(\omega)E_{\mu}^{(+)}E_{\nu}^{(-)}
\end{equation}
%
where $\mathbf{E}^{(\pm)}$ are the possitive/negative frequency components of the field. $\alpha_{\mu\nu}(\omega)$ can be found by looking at the (time averaged) shift in the energy of the a given state state using second order time-dependent perturbation theory~\cite{SteckTextbook,deutsch_quantum_2010}. For the ground state $\ket{g}$ the polarizability takes the form
%
\begin{equation}
	\alpha_{\mu\nu}(\omega)=\sum_j\frac{2\omega_{jg}\bra{g}d_\mu\ket{e_j}\bra{e_j}d_\nu\ket{e_j}}{\hbar(\omega_{jg}^2-\omega^2)},
\end{equation}
%
where $\ket{e_j}$ represent the excited states and $\omega_{jg}=(E_j-E_g)/\hbar$. The polarizability is a rank-2 tensor and can be written in terms of it's irreducible tensor operators\footnote{A collection of operators that transforms under
rotations like the spherical harmonics $Y_{kq}(\theta, \phi)$} (see~\cite{SteckTextbook} for a complete derivation). In the limit of small magnetic fields so that $F$ and $m_F$ are good quantum numbers describing the state of the atom $\ket{n, F, m_F}$ the dipole Hamiltonian in this representation takes a convenient form
%
\begin{align}
	\hat{H}_{\rm{dip}}= &\alpha^{(0)}(\omega)(\mathbf{E}^{(-)}\cdot\mathbf{E}^{(+)}) 
	+\alpha^{(1)}(\mathbf{E}^{(-)}\times\mathbf{E}^{(+)})\cdot\mathbf{\hat{F}}  \nonumber \\ 
	&+ \alpha^{(2)}E_i^{(-)}E_j^{(+)}	\left(\frac{1}{2}(F_iF_j+F_jF_i)-\frac{1}{3}\mathbf{\hat F}^2\delta_{i,j}\right)\Big],
\end{align}
%
where $\alpha^{(0)}$, $\alpha^{(1)}$ and $\alpha^{(2)}$ are the scalar, vector and tensor polarizabilities respectively and $\hat{\mathbf{F}}$ is the total angular momentum operator. For all our experiments $\alpha^{(2)}$ is very small so I will limit the discussion to the effect of the first two terms. The scalar term is responsible for the dipole force that allow us to trap atoms using off-resonant light and the vector component is necessary for engineering spin-orbit coupling through Raman transitions and other spin-dependent potentials. 

\subsection{Scalar polarizability}

The scalar polarizability takes the form
%
\begin{equation}
	\alpha^{(0)}=\sum_j\frac{2\omega_{jg}\bra{g}\mathbf{d}\cdot\hat{\epsilon}\ket{e_j}\vert^2}{\hbar(\omega_{jg}^2-\omega^2)},
\end{equation}
%
where $\hat{\epsilon}$ represents the polarization vector of the light. The matrix element can be expressed in terms of the Clebsch-Gordan coefficients and the reduced matrix element using the Wigner-Eckart theorem. For the ground state of an Alkali atom ($J=1/2$) the expression above gets simplified to
%
\begin{equation}
	\alpha^{(0)}\approx\sum_{J'}\frac{2\omega_{JJ'}\vert\langle J=1/2 \| \mathbf{d}\|J'\rangle\vert^2}{3\hbar(\omega_{JJ'}^2-\omega^2)}.
\end{equation}

The dipole matrix elements needed to compute the polarizability are related to the transition scattering rate via Fermi's golden rule~\cite{Sakurai,SteckTextbook}
\begin{equation}
	\Gamma_{JJ'}=\frac{\omega_{JJ'}^2}{3\pi\epsilon_0\hbar c^3}\frac{2J+1}{2J'+1}\vert\langle J \| \mathbf{d}\|J'\rangle\vert^2,
\end{equation}
%
and combining this with the expression for the intensity of the electric field $I(\r)=2\epsilon_0c\vert \mathbf{E}(\r)\vert^2$ we find that the energy of the ground state manifold is shifted by
\begin{equation}
	U(\omega,\r)=-\frac{\pi c^2 I(\r)}{2}\left[ \frac{2\Gamma_{\rm{D1}}}{\omega_{\rm{D1}}^2}\left(\frac{1}{\omega+\omega_{\rm{D1}}}-\frac{1}{\omega-\omega_{\rm{D1}}}\right)+\frac{2\Gamma_{\rm{D2}}}{\omega_{\rm{D2}}^2}\left(\frac{1}{\omega+\omega_{\rm{D2}}}-\frac{1}{\omega-\omega_{\rm{D2}}}\right)\right],
\end{equation}
%
where only the contribution of the closest transitions from the D1 and D2 lines are included. Here $U(\r)$ is related to the real part of the polarizability which is in fact a complex valued number. So far I have only considered the real part by assuming the excited states have an infinitely long lifetime. However, in reality the atom can spontaneously emit photons and decay. This exponential decay can be accounted for by adding an imaginary contribution to the energies $\omega_D\rightarrow\omega_D+i\Gamma_D\omega^3/\omega_D^3$ of the D1 and D2 transitions~\cite{grimm_optical_2000}. The scattering rate is related to the imaginary part of the polarizability and is given by
%
\begin{equation}
	\Gamma(\omega,\r)=\frac{\pi c^2I(\r)}{2\hbar}\left[ \frac{2\Gamma_{\rm{D1}}\omega^3}{\omega_{\rm{D1}}^6}\left(\frac{1}{\omega+\omega_{\rm{D1}}}-\frac{1}{\omega-\omega_{\rm{D1}}}\right)^2+\frac{2\Gamma_{\rm{D2}}\omega^3}{\omega_{\rm{D2}}^6}\left(\frac{1}{\omega+\omega_{\rm{D2}}}-\frac{1}{\omega-\omega_{\rm{D2}}}\right)^2\right]
\end{equation}

The energy shift $U(\omega,\r)$ is a conservative term and is related to dipole trapping while the scattering term $\Gamma(\omega,\r)$ is dissipative and is important for laser cooling. In the context of engineering potentials for ultracold atoms with off-resonant light, the scattering is translated into heating because every time an atom emits a photon with angular frequency $\omega_L$ it gets a recoil momentum $\hbar\k_L$.  

\subsubsection{Optical trapping}
One important application of the scalar light-shift is to create optical traps for our clouds of ultracold atoms. If we use optical fields with non-uniform spatial intensity we can generate traps (and anti-traps) because the atoms will experience a force $F_{\rm{dip}}=-\nabla U(\r)$. Generally for the optical frequencies we use in the lab $\omega+\omega_{\rm D}\gg\omega-\omega_{\rm D}$ so the rotating wave approximation is valid and we can neglect the terms proportional to $1/(\omega+\omega_{\rm D}$ so the sign of the detuning 

or a Gaussian beam with intensity



% the scalar polarizability takes the form 
% The matrix element in Equation~\ref{eq:scalar_pol} can be expressed using the Wigner-Eckart theorem~\cite{Sakurai} in terms of the Clebsch-Gordan coefficients and the reduced matrix elements. 
%Here I will only consider the case of oscillating electric fields $\mathbf{E}=E_0\cos(\omega t)\boldsymbol{\epsilon}$ (i.e. plane waves of electromagnetic radiation) which are relevant to our experiments. 

write tensor polarizability, write it into irreducible components. Mention that tensor component is very small and ignored. 

Scalar polarizability: real part is light shift, complex part is scattering. Apply rwa and get the classic expressions for trapping potential and scattering. Make plot of polarizabily for RB87 Sub subsection: dipole traps

Tensor polarizability



Can break interaction into scalar, vector and tensor part. Interaction can also be resonant or off-resonant.

\note{Why do you only get second order perturbation theory effects? I think it has something to do with unperturbed atomic states being eigenstates of the parity operator}


\subsection{Vector light shift: Raman coupling}

The geometry and wavelength of the Raman fields determine the natural units of the system: the single photon recoil momentum $k_{\mathrm{L}}=\sqrt{2}\pi/\lambda_{\mathrm{R}}$ and its associated recoil energy $E_{\mathrm{L}}=\hbar^2k_{\mathrm{L}}^2/2m$, as well as the direction of the recoil momentum $\mathbf{k}_{\mathrm{L}}=k_{\mathrm{L}}\ex$. The Raman wavelength was $\lambda_{\mathrm{R}}=790.032\,\nm$, as usual, so that the scalar light shift is zero. 

\section{Rashba SOC in condensed matter}

% In earlier sections we discussed the spin-orbit coupling interaction between spin and angular momentum that gives rise to the atomic level structure. In condensed matter systems, there is another kind of spin-orbit coupling that links the spin of the electrons with the linear or crystal momentum. In 2D materials, SOC can be represented as a sum of Rashba~\cite{bychkov_oscillatory_1984} and Dresselhaus~\cite{dresselhaus_spin-orbit_1955} SOC. 


\note{Something about what we use this transitions for and how we usually ignore all other levels. (???)}

\note{TODO: what about matrix elements between -1 and +1?. Move to intro chapter}


\section{Magnetic interaction}
\subsection{Static magnetic fields}
Also talk about the Paschen–Back effect occurs in a strong external magnetic field. The spin and orbital angular momen- tum precess independently about the magnetic field.

Uniform fields: Zeeman splitting, Breit-Rabi formula
Gradients: Quadrupole potentials and Stern-Gerlach

\subsection{Oscillatory magnetic fields}
\label{seq:rf_coupling}
RF coupling and microwave coupling



\section{Applications}
\subsection{Quantum coherent dynamics?}
\subsection{Adiabatic rapid pasage}
\label{sec:arp}
\subsection{The Rabi cycle}
\subsection{Ramsey interferometry}



\section{Floquet theory}



\section{Absorption imaging}
\subsection{Time of flight imaging}
\subsection{Partial transfer absorption imaging: magnetic field stabilization}
\label{sec:ptai}
We then apply a pair of $250\,\mu\mathrm{s}$ microwave  pulses that each transfer a small fraction of atoms into the $5^2{\rm S}_{1/2}$ $f=2$ manifold that we use to monitor and stabilize the bias field \cite{leblanc_direct_2013}. The microwave pulses are detuned by $\pm 2\, \kHz$ from the $\ket{f=1,m_F=0}\leftrightarrow\ket{f=2,m_F=1}$ transition and spaced in time by $33\, \mathrm{ms}$ (two periods of $60\, \mathrm{Hz}$). We imaged the transferred atoms following each pulse using absorption imaging\footnote{We did not apply repump light during this imaging, so the untransferred atoms in the $f=1$ manifold were largely undisturbed by the imaging process.}, and count the total number of atoms $n_1$ and $n_2$ transferred by each pulse. The imbalance in these atom numbers $(n_1-n_2)/(n_1+n_2)$ leads to a $4\kHz$ wide error signal that we use both to monitor the magnetic field before each spectroscopy measurement and cancel longterm drifts in the field. 
\section{Floquet}
\label{sec:Floquet_theory}
How to treat systems when RWA is not valid and how to create new effective (stroboscopic) Hamiltonians.

%%%%%%%%%%%%%%%%%%%%%%%%%%%%%%%%%%%%%%%%%%%%%%%%%%%%%%%%%%%%%%%
%
%Graveyard
%
%%%%%%%%%%%%%%%%%%%%%%%%%%%%%%%%%%%%%%%%%%%%%%%%%%%%%%%%%%%%%

% The eigenstate of the perturbed Hamiltonian are linear combinations of the unperturbed eigenstates $\ket{n}$  
% %
% \begin{equation}
% 	\ket{\psi}=\sum_n a_n(t)e^{-iE_n t/\hbar}\ket{n},
% \end{equation}
% %
% using the time-dependent Schr\"odinger equation we can find equations for the coefficients $a_n(t)$
% %
% \begin{equation}
% 	i\hbar \partial_ta_n(t)=\sum_k\bra{n}\hat{H}_{\rm{dip}}\ket{k}a_k(t)e^{i\omega_{n,k}t}
% \end{equation}
% %
% where $\omega_{nk}=(E_n-E_k)/\hbar$. If we consider the perturbation being turned on at $t=0$ and if $\omega\neq\omega_{nk}$ the first order coefficient is
% %
% \begin{equation}
% 	a_n^{(1)}=-\frac{\bra{n}d_iE_0\ket{m}}{2\hbar}
% \end{equation}

% For $S$ electrons the hyperfine splitting can be described by the Hamiltonian $\hat H_{\rm{hfs}} = A_{\rm{hfs}}\mathbf{I}\cdot\mathbf{J}$\footnote{Notice how both the fine and hyperfine structure arise from a spin-orbit coupling interaction, we will discuss a very different type of spin-orbit coupling in future chapter.}. Figure~\ref{fig:fs_hfs}c shows the fine structure getting further split into states of total angular momentum $\mathbf{F}=\mathbf{J}+\mathbf{I}$. $\Rb87$ has a nuclear spin $I=3/2$ and therefore its ground state hyperfine configuration has $F=1$ and $F=2$. Here
% ~\cite{Steck} 

% Lets now consider the example of an atom in the presence of an oscillating electric field with amplitude $E_0$ and polarization $\epsilon$ $\mathbf{E}=E_0\cos(\omega t)\boldsymbol{\epsilon}$. We will use time-dependent perturbation theory to calculate the resulting energy shifts. 


% So far I have considered that the excited state has an infinitely long lifetime and does not decay. However, in reality the atom can spontaneously emit photons and decay. The lifetime of the excited state is given by $1/\Gamma_{e_j}$. The exponential decay of the excited state corresponds to adding a imaginary contribution to the energy $E_{e_j}\rightarrow E_{e_j}-i\hbar\Gamma_{e_j}/2$ which makes the polarizability a complex valued number. 

% The decay rate of a transition is related to the square of the dipole matrix element via Fermi's golden rule~\cite{Sakurai}

% \begin{equation}
% 	\Gamma_{J_gJ_e}=\frac{\omega_0^2}{3\pi\epsilon_0\hbar c^3}\frac{2J_g+1}{2J_e+1}\vert\langle g \| \mathbf{d}\|e\rangle\vert^2
% \end{equation}

% where $\vert\langle g \| \mathbf{d}\|e\rangle\vert^2$ is a reduced matrix element
% I will now look at the scalar polarizability term for Alkali atoms. For the case of far-detuned light such that the hyperfine $F$ levels are not well resolved the scalar polarizability can be calculated using the fine structure $J$ states. For an atom in an initial state $J$

% use scattering rate to find reduced matrix element write final expresion. skip the j, go back to generic notation
% %
% \begin{equation}
%  	\alpha^{(0)}\approx\sum_{J'}\frac{\vert\langle J \| \mathbf{d}\|J'\rangle \vert^2}{3\hbar}\left(\frac{1}{\omega+\omega_{JJ'}}-\frac{1}{\omega-\omega_{JJ'}}\right)
%  \end{equation} 
% If we consider Alkali atoms in the ground state hyperfine manifold in the presence of far detuned light such that the hyperfine levels are not well resolved we need to calculate the matrix elements using the $J$ states

