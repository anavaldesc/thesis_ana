% !TEX root = mainthesis.tex
% Informs the editor to look one folder up for the main file.
%Chapter 1

\renewcommand{\thechapter}{1}

\chapter{Introduction}

Bose-Einstein condensation in dilute atomic gases was observed for the first time in 1995 

It has been more than 20 years since the first observation of Bose-Einstein condensation in dilute atomic gases


Quantum degenerate gases have proven to be an excellent platform for the analog simulation of complex quantum systems.

Topological matter

, a test quantum system, here ultracold atoms, is used to simulate a more
complicated, less experimentally accessible quantum system, such as a non-trivial
material from condensed matter physics. In order to get to the point where unsolved
problems can be solved with quantum simulation, tools must be built up to create
and verify Hamiltonians in the test system that are relevant to the more complex
target system

single-particle dispersions that are analogs to those present in condensed matter systems, thereby creating exotic atomic 'materials', with interaction-dominated or topologically non-trivial band structures

Why is quantum simulation important:

Ultracold atoms systems could potentially enable the simulation of 

\begin{itemize}
	\item Can help understand problems that are not easy to solve numerically or analytically. High temperature superconductors, frustrated systems, as good examples.
	\item Create analogues to systems that would otherwise not be possible to study. Example: the expanding universe, Hofstadter at large magnetic fields.
	\item 
\end{itemize}


Additionally, such systems allow us to create new exotic atomic `materials' in the lab that can help deepen our understanding of certain physical phenomena such as the role of interactions and topology in band structures. 

This thesis focuses on the development of new tools for the characterization and control of engineered quantum systems. We deployed a Fourier transform spectroscopy technique which allows us to probe the spectrum of a system by only looking at its quantum coherent evolution



 and applies them 




ur main result shows the quantum engineering of a system with Rashba-like~\cite{bychkov_oscillatory_1984} spin-orbit coupling. Spin-orbit coupling in two-dimensional materials plays an important role in the quantum spin Hall effect and the realization of topological insulators. Unlike conventional materials, we can engineer the Rashba interaction without an underlying crystalline structure which gives rise to unconventional topology characterized by non-integer valued invariants. 


Cold atoms for engineering topological matter. 

The scope of this thesis mostly falls under the latter point. 

In order for new advances in the field of quantum simulati 

Our system 

relies only on the unitary evolution of an initial state suddenly subjected t


In particular, the properties of 



systems that do not exist in nature but can help us learn or understand something... or are just fun!
That have no physical analogues that can help deepen our understanding of certain physical properties. This thesis falls mostly under the third point. 






Don't forget to talk about topology! It starts with condensed matter but has been relevant to many other systems. Many Noble prizes awarded, many applications and potential applications found. 

Start with topology and move into quantum simulation? Or the other way around?


\section{Thesis overview}

In Chapters 2 and 3 I will describe the basic theory of Bose-Einstein condensation and the technical details of our experimental apparatus that produces $\Rb87$ BECs. In Chapter 4 I will describe our quantum simulation toolkit, the standard techniques that we use to manipulate and detect ensembles of ultracold atoms that are necessary for all of our experiments. Chapter 5 describes a Fourier transform spectroscopy technique that exploits the relation between quantum coherent evolution and the underlying spectrum of a system and that was used to characterize experiments described later in the thesis. Chapter 6 describes an implementation of continuous dynamical decoupling that helped to both make our system more robust against environmental noise and also allowed us to couple the internal states of the atoms in new ways that were not possible before, opening the path for new kinds of quantum simulations described in Chapters 7 and 8. In Chapter 8 I describe the experimental realization of Rashba spin-orbit coupling for a quantum system without a crystalline structure and has unconventional topology characterized by non-integer topological invariants. Finally, Chapter 8 describes the experimental implementation of a fractional period adiabatic superlattice, an intermediate step necessary for us to generate Hofstadter cylinders with non-zero magnetic flux in the future. 

Apendix are experiments that I contributed to but are not included in the thesis. Also things related to new apparatus?

% This is the story I'm telling: Cold atoms are a great platform. They are very controllable and 

% \begin{itemize}
% 	\item A toolbox for manipulation and engineering of the system
% 	\item Characterizing the system, here would go absorption imaging (TOF and SG included), maybe PTAI. Quantum coherent evolution to learn about a system: Rabi spectroscopy, Ramsey spectroscopy
% 	\item Fourier transform spectroscopy.
% 	\item Combining the toolbox and characterization techniques 
% 	%\subitem Is this a thing?	
% \end{itemize}