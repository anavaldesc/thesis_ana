% !TEX root = mainthesis.tex
% Informs the editor to look one folder up for the main file.
%Chapter 1

\renewcommand{\thechapter}{1}

\chapter{Introduction}



Why is quantum simulation important:

\begin{itemize}
	\item Can help understand problems that are not easy to solve numerically or analytically. High temperature superconductors, frustrated systems, as good examples.
	\item Create analogues to systems that would otherwise not be possible to study. Example: the expanding universe, Hofstadter at large magnetic fields.
	\item Create new exotic systems that do not exist in nature but can help us learn or understand something... or are just fun!
\end{itemize}

Don't forget to talk about topology! It starts with condensed matter but has been relevant to many other systems. Many Noble prizes awarded, many applications and potential applications found. 

Start with topology and move into quantum simulation? Or the other way around?

\section{Thesis overview}

In Chapters 2 and 3 I will describe the basic theory of Bose-Einstein condensation and the technical details of our experimental apparatus that produces $\Rb87$ BECs. In Chapter 4 I will describe our quantum simulation toolkit, the standard techniques that we use to manipulate and detect ensembles of ultracold atoms that are necessary for all of our experiments. Chapter 5 describes a Fourier transform spectroscopy technique that exploits the relation between quantum coherent evolution and the underlying spectrum of a system and that was used to characterize experiments described later in the thesis. Chapter 6 describes an implementation of continuous dynamical decoupling that helped to both make our system more robust against environmental noise and also allowed us to couple the internal states of the atoms in new ways that were not possible before, opening the path for new kinds of quantum simulations described in Chapters 7 and 8. In Chapter 8 I describe the experimental realization of Rashba spin-orbit coupling for a quantum system without a crystalline structure and has unconventional topology characterized by non-integer topological invariants. Finally, Chapter 8 describes the experimental implementation of a fractional period adiabatic superlattice, an intermediate step necessary for us to generate Hofstadter cylinders with non-zero magnetic flux in the future. 

Apendix are experiments that I contributed to but are not included in the thesis. Also things related to new apparatus?

% This is the story I'm telling: Cold atoms are a great platform. They are very controllable and 

% \begin{itemize}
% 	\item A toolbox for manipulation and engineering of the system
% 	\item Characterizing the system, here would go absorption imaging (TOF and SG included), maybe PTAI. Quantum coherent evolution to learn about a system: Rabi spectroscopy, Ramsey spectroscopy
% 	\item Fourier transform spectroscopy.
% 	\item Combining the toolbox and characterization techniques 
% 	%\subitem Is this a thing?	
% \end{itemize}