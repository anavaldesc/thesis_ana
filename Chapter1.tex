% !TEX root = mainthesis.tex
% Informs the editor to look one folder up for the main file.
%Chapter 1

\renewcommand{\thechapter}{1}

\chapter{Introduction}

Bose statistics were first developed in 1924 to describe the quantum behavior of photons \cite{bose_plancks_1924}, quanta of light with integer-valued spin. Einstein generalized this work to include massive particles~\cite{einstein_quantentheorie_2006} and predicted the macroscopic occupation of the ground state bellow a critical temperature. Today, we routinely produce atomic Bose-Einstein condensates (BECs) in the laboratory and even use them as a platform for the analog simulation of complex quantum systems. The field has certainly come a long way! 

A connection between Bose-Einstein condensation and the onset of superfluidity in liquid $^4$He was first made in 1938~\cite{london_bose-einstein_1938}. However, due to the strong interactions, the occupancy of the lowest energy state is dramatically reduced, leading to the search of this phase for weakly interacting Bose gases. Bose-Einstein condensation in dilute atomic gases was observed for the first time in 1995 in vapors of $\Rb87$~\cite{anderson_observation_1995}, $^{23}$Na~\cite{davis_bose-einstein_1995} and $^7$Li~\cite{bradley_evidence_1995}. The dilute nature of this gases required cooling down to temperatures never achieved before and it wasn't until experimental techniques of laser cooling and trapping were developed that bosonic atoms were cooled the critical condensation temperature\footnote{For a more in depth story of the BEC field I highly recommend reading~\cite{ketterle_w._making_1999}.}. The experimental realization of this new phase of matter opened new possibilities for studying macroscopic quantum phenomena such as the propagation of collective modes~\cite{jin_collective_1996,mewes_collective_1996} and interference of coherent matter waves~\cite{andrews_observation_1997} and jump-started an ever-growing field of research. For this achievement, Eric Cornell, Carl Wieman, and Wolfgang Ketterle were awarded the 2001 Nobel Prize in physics.

It has been more than 20 years (and many more in the learning of atomic physics) and the field of quantum degenerate gases has expanded to include degenerate Fermi gases~\cite{demarco_onset_1999} and spinor gases~\cite{stamper-kurn_spinor_2013}. As the field continues to grow, new control and detection techniques are being constantly developed, enabling the use of BECs and ultracold atomic systems in general not just as an object of study by themselves but as tools for a wide range of scientific endeavors, from precision measurements~\cite{zhang_precision_2016} to the analog simulation of complex systems.

Quantum degenerate gases have proven to be an ideal platform for quantum simulation~\cite{bloch_quantum_2012}. A very straightforward example comes from the use of optical lattices, where the periodic potential imparted by standing waves of light serves as an analogue to the crystal structure in a solid. Perhaps the first iconic realization of a quantum simulation was the study of the Bose-Hubbard model in three-dimensional optical lattices~\cite{greiner_quantum_2002}, the bosonic analogue of a model which is believed to be relevant to high-$T_c$ superconductors. 

The development of light-induced gauge-fields~\cite{goldman_light-induced_2014} has been another important milestone in the field of quantum simulation. Such fields can be used to mimic the effect of magnetic~\cite{synthetic_dimensions_theory,lin_synthetic_2009} and electric~\cite{lin_synthetic_2011} fields with potential applications to the realization of quantum Hall materials with large magnetic fluxes~\cite{synthetic_dimensions_theory,miyake_realizing_2013} and fractional quantum Hall states~\cite{cooper_reaching_2013}. Furthermore, light-induced gauge fields can be used to engineer spin-orbit coupling interactions~\cite{galitski_spin-orbit_2013} as those present in two-dimensional materials, a necessary ingredient for the spin quantum Hall effect and certain kinds of topological insulators~\cite{hasan_colloquium:_2010}. Chapter~\ref{ch:Rashba} will focus on a new implementation of Rashba-type spin-orbit~\cite{bychkov_oscillatory_1984,campbell_rashba_2016} coupling in an ultracold $\Rb87$ atoms. 

The precise level of control and tunability of ultracold atomic systems allow us to readily implement important physical models in the laboratory. Furthermore, we can go beyond conventional materials existing in nature, and thereby create new exotic atomic `materials', with interaction-dominated or topologically non-trivial band structures that can help deepen our understanding of the physical consequences of these effects on materials.

This thesis focuses on the development of new tools for the characterization and control of engineered quantum  and applies them to create and characterize a topological system with Rashba-type spin-orbit coupling. 

The creation of new engineered materials requires the ability to characterize their single particle energies. We developed a Fourier transform spectroscopy technique which allows us to probe the single particle spectrum, and thereby verify our quantum engineering, by only looking at quantum coherent evolution. 

Atomic systems are susceptible to environmental noise, leading to undesired effects such as the loss of coherence. In particular laboratories such as ours greatly suffer from noise in ambient magnetic fields and go through great efforts to diminish their effects. We implemented continuous dynamical decoupling (CDD) on a set of internal atomic states which renders them first order insensitive magnetic field changes, effectively turning them into clock states. These CDD states are not just a robust basis for performing experiments, they additionally gave us access to new matrix elements which were essential for the engineering of the Rashba Hamiltonian as well as other novel lattice systems~\cite{anderson_realization_2019} not presented in this thesis. 

The engineering of Rashba spin-orbit coupling is certainly condensed matter inspired. However, part of the beauty of engineered quantum systems is we can depart from conventional materials, for example, by considering a system with Rashba SOC but without an underlying crystalline structure. We do so in the last part of this thesis and find that our conventional understanding of the topology of Bloch bands is defied by measuring half-integer valued topological invariants. 

\section{Thesis overview}

This thesis describes both the standard experimental control and measurement techniques used to create BECs of $\Rb87$ as well as new techniques developed in our lab that were applied to the engineering of Rashba spin-orbit coupling. 

Chapter~\ref{ch:BECs} describes the basic theory of Bose-Einstein condensation in dilute gases. I focus on the properties of gases confined to harmonic potentials and their density and momentum distributions as they are most relevant to the experiments presented here.

Chapter~\ref{ch:Ch3} describes the basic properties of Alkali atoms and their interactions with magnetic and electric fields which are used as standard tools for the creation, manipulation and detection of ultracold atomic systems.

Chapter~\ref{ch:RbLi} summarizes the experimental apparatus where all the experiments were performed. It additionally mentions the most important upgrades to the apparatus that were not reported previously.

Chapter~\ref{ch:Fourier_spectroscopy} describes a Fourier transform spectroscopy technique that exploits the relation between quantum coherent evolution and the underlying spectrum of a system and that was used to characterize experiments described later in the thesis.

Chapter~\ref{ch:clock_states} describes an implementation of continuous dynamical decoupling using a strong radio-frequency magnetic field that helped to both make our system more robust against environmental magnetic field noise and also allowed us to couple the internal states of the atoms in new ways that were not possible before, opening the path for new kinds of quantum simulations described in Chapters 8. 

Chapter~\ref{ch:Topology} presents basic concepts of topology in physics and its application to the band theory of solids. These concepts will be important for a better understanding of the topological properties of our Rashba spin-orbit coupled system. 

Chapter~\ref{ch:Rashba} I describes a new experimental realization of Rashba spin-orbit coupling using a combination of laser beams that couple a set of CDD states. The system described in this chapter has a topological dispersion relation but no underlying crystalline structure which allows for topological invariants to take non-integer values. 

Appendix~\ref{app:RbLi} summarizes the best and worst aspects of the experimental apparatus as a guide for future generations working in BEC labs.

Appendix~\ref{app:new_apparatus} describes my work related to the design and construction of a dual species apparatus for the production of BECs of $\Rb87$ and $^{39}$K.

Appendix~\ref{app:Rashba_derivation} shows the derivation of the full time-dependent Hamiltonian describing the Raman dressing of the RF dressed states (Chapter~\ref{ch:clock_states}) used to generate Rashba-type spin-orbit coupling (Chapter~\ref{ch:Rashba}).


% Apendix are experiments that I contributed to but are not included in the thesis. Also things related to new apparatus?



% A great deal of current research focuses on understanding
% the physical consequences of these new materials, and
% experimental studies of topological insulators and superconductors in solid state systems continue apace. Furthermore,
% there is significant activity in exploring the nature of the
% strongly correlated phases of matter that arise in these
% materials, notably to construct strong-correlation variants of
% these topological states of weakly interacting electrons.
% Theory suggests many interesting possibilities, which are still
% seeking experimental realization and verification.
% Such questions are ideally addressed using realizations with
% cold atomic gases. Cold atomic gases allow strongly interacting phases of matter to be explored in controlled experimental
% settings. However, a prerequisite for quantum simulations of
% such issues is the ability to generate topological energy bands
% for cold atoms. This poses a significant challenge, even at this
% single-particle level. Realizing topological energy bands
% typically requires either the introduction of effective orbital
% magnetic fields acting on neutral atoms and/or the introduction of a spin-orbit coupling between the internal spin states of
% an atom and its center-of-mass motion. This is an area of
% research that has attracted significant attention over the last
% years, both theoretical and experimental. Much progress has
% been made in developing techniques to generate artificial
% magnetic fields and spin-orbit coupling for neutral atoms
% (Zhai, 2015; Dalibard, 2016; Aidelsburger, Nascimbene, and
% Goldman, 2017). The use of these techniques in the setting of
% optical lattices has led to the realization and characterization
% of topological Bloch bands for cold atoms

% \section{Graveyard}
% high degree of control and tunability 







% systems that do not exist in nature but can help us learn or understand something... or are just fun!
% That have no physical analogues that can help deepen our understanding of certain physical properties. This thesis falls mostly under the third point. 



%  BECs `in the wild' 




% Topological matter

% , a test quantum system, here ultracold atoms, is used to simulate a more
% complicated, less experimentally accessible quantum system, such as a non-trivial
% material from condensed matter physics. In order to get to the point where unsolved
% problems can be solved with quantum simulation, tools must be built up to create
% and verify Hamiltonians in the test system that are relevant to the more complex
% target system

% single-particle dispersions that are analogs to those present in condensed matter systems, thereby creating exotic atomic 'materials', with interaction-dominated or topologically non-trivial band structures

% Why is quantum simulation important:

% Ultracold atoms systems could potentially enable the simulation of 

% \begin{itemize}
% 	\item Can help understand problems that are not easy to solve numerically or analytically. High temperature superconductors, frustrated systems, as good examples.
% 	\item Create analogues to systems that would otherwise not be possible to study. Example: the expanding universe, Hofstadter at large magnetic fields.
% 	\item 
% \end{itemize}


