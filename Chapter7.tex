% !TEX root = mainthesis.tex
%Chapter 7

\renewcommand{\thechapter}{7}

\chapter{Rashba spin-orbit coupling}

\section{Review of topology}

\subsection{Berry's phase and curvature, Chern numbers, etc.}

\subsubsection{Example: topology of a 2D electron gas}
A generic Hamiltonian for a 2D electron gas is 
%
\begin{equation}
	\hat{H}(\mathbf{k})= \mathbf{h}(\mathbf{k})\cdot\vec{\sigma}
	\label{eq:2D_Hamiltonian}
\end{equation}
%
where $\vec{\sigma}=(\sigma_x, \sigma_y, \sigma_z)$ are the Pauli matrices and $\mathbf{h}(\mathbf{k})$
\subsection{A brief history of topology in physics}
Topological phase transitions vs. regular phase transitions?
The 2016 Nobel prize in physics was awarded to David J. Thoules, F. Duncan M. Haldane and J. Michael Kosterlitz for theoretical discoveries of topological phase transitions and topological phases of matter. Kosterlitz and Thoules first used topology to describe phase transition on 2D materials (1973, cite Long Range Order and Metastability in Two Dimensional Solids and Superfluids). A decade later Thoules, Kohomoto, Nightingale, and den Nijs explain used the topology arguments to explain the Quantum Hall effect (Quantized Hall Conductance in a Two-Dimensional Periodic Potential). In their paper they showed that the quantization of the Hall conductivity
%
\begin{equation}
	\sigma_{xy} = Ne^2/h
	\label{eq:hall_conductivity}
\end{equation}
%
is determined by the underlying topology of the band structure (how much can the Hamiltonian be deformed without closing an energy gap). N is equal to the Chern number. 



\section{Rashba spin-orbit coupling in solids}
\subsection{Dirac points}
Dirac points occur in band structures when there is inversion ($\mathcal{P}$) and time reversal ($\mathcal{T}$) symmetry.  \mathcal{P} symmetry takes and therefore $h_z(\mathbf{k})=0$ and for small crystal momentum $\mathbf{q}$ the Hamiltonian\ref{eq:2D_Hamiltonian} resembles that of a massles Dirac fermion $\hat H(\mathbf k)=\hbar v_F\mathbf q\cdot \vec \sigma$, where $v_F$ is a velocity. The $\mathcal{T}$ symmetry can be broken for example by applying a magnetic field, in which case the degeneracy at the Dirac point is broken and \ref{eq:2D_Hamiltonian} becomes a massive Dirac Hamiltonian. 

Something about quantum spin Hall effect and spin orbit coupling?
\section{Implementation of the Rashba Hamiltonian with cold atoms}

\section{Study: experimental autocorrelation function}


