% !TEX root = mainthesis.tex
%Chapter 9

\renewcommand{\thechapter}{9}


\chapter{Conclusions and outlook}

This thesis presented new experimental techniques which have proven to be useful in the control and characterization of ultracold atomic systems and applied them in a new implementation Rashba-type spin orbit coupling.

We developed a Fourier stransform spectroscopy technique~\cite{valdes-curiel_fourier_2017} which is based on measuring the quantum coherent evolution of a single particle system under a quench of a Hamiltonian of interest. This technique was successfully used to measure the dispersion relation of a BEC with tunable 1D (equal combination of Rashba and Dresselhaus) spin-orbit coupling. The use of this technique was extended to thermal gases with broad momentum distributions to perform a parallelized measurement of the dispersion relation of a system with Rashba-type SOC~\cite{valdes-curiel_unconventional_2019} as well as the band structure of a fractional period adiabatic superlattice~\cite{anderson_realization_2019}. 

We implemented CDD ground hyperfine manifold of $\Rb87$ by applying a strong RF magnetic field~\cite{trypogeorgos_synthetic_2018}. The CDD states are first order insensitive to magnetic field fluctuations, making them effective clock states, and additionally have non-zero matrix coupling elements which allows for cyclical couplings that are not possible in the bare hyperfine $\ket{m_F}$ states. The clock states have made our experiments more robust against magnetic field noise and were a necessary ingredient for the implementation of Rashba spin-orbit coupling as well as the engineering of fractional period adiabatic superlattice and an ongoing project involving Hofstadter cylinders.

Finally we show a new implementation of Rashba spin-orbit coupling using Raman induced transitions of the CDD states and characterize the topology underlying this system.  We present a protocol for performing quantum state tomography which involves a three-arm Ramsey like interferometer and use it to reconstruct the momentum-dependent wave function and calculate topological invariants. Unlike conventional materials with an underlying crystalline structure where topological invariants take integer values, we find that our system in the continuum is characterized by half-integer invariants. Our Rashba implementation offers the possibility of studying new ground state physics at the nearly degenerate minima like the formation of fragmented condensates or possible realizations of fractional Hall like states. One open question lead for a half-integer Chern number system like ours is what kind of edge states emerge at the interfaces where the Chern numbers differ by a non-integer number. 




