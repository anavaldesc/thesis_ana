%Preface

\renewcommand{\baselinestretch}{2}
\small\normalsize
\hbox{\ }
 
\vspace{-.65in}

\begin{center}
\large{Preface} 
\end{center} 

A few weeks after I had started writing this thesis Ian came to me and asked if I had learned all the physics I wish I had known while I was doing my PhD. I felt a bit puzzled. I had been operating the lab, analyzing data and writing papers for a while, of course I already knew the physics relevant to my research! It finally hit me when I was writing the introductory chapters how many subtleties I had missed and how much I still didn't know. As experimental physicists being in the lab can give us some physical intuition and a sense of understanding but sometimes that is not enough. This was a very striking and unexpected side effect of the thesis and I invite experimentalists reading this to challenge their lab intuition. 

In the end, I actually found it very enjoyable to look back at the history of our field and to put my research into a bigger context. I never thought the words thesis and  enjoyable could go well together! My advice for a graduate student starting to write a thesis is try to enjoy the ride, it will definitely be stressful and overwhelming but try to make the best of it. 

Finally, I would like to mention that I wrote a lot of this thesis with new students in the lab in mind hoping it will be a useful reference in their research.