% !TEX root = mainthesis.tex
%Chapter 4

\renewcommand{\thechapter}{4}

\chapter{Manipulation and detection of ultra-cold atoms}

All of our experiments rely on the interaction of atoms with electric and magnetic fields, both for the preparation of ultra-cold atoms through laser cooling and trapping, for the engineering of interesting Hamiltonians, and for detection. In this chapter I will first describe the electronic structure of $\Rb87$ which makes all of our experiments possible and then I will review the effects of the magnetic and electric interactions that are relevant to our experiments. I will not cover laser cooling which has been covered extensively in the literature (see~\cite{metcalf_deceleration_1999} for example). 

\section{Electronic structure of $^{87}$Rb}



\section{Atomic interactions with electromagnetic fields}

Rb is an Alkali metal (also Li, which exists in our vacuum chamber but was never used). Alkali metals correspond to the first group (leftmost column) of the periodic table and are characterized by having a single valence electron, which makes the description of their internal structure much simpler than that of other elements. In genera we can describe the state of an electron in an atom by it's angular momentum $\mathbf L$ and its spin $\mathbf{S}$. Because of Pauli's exclusion principle there can not be two electrons with the same quantum numbers and in multi-electron atoms they tend to fill `shells' of different angular momentum values, historically labeled by the letters $S,\ P,\ D,\ F,\ ...$\footnote{This terms were used to describe the lines in the emission spectra when they were first discovered. $S$ stands for sharp, $P$ for principal $D$ for diffuse and $F$ for further noted} (corresponding to $\mathbf{L}=1,\ 2,\ 3,\ 4,\ ...$). In particular Rb has 4 filled shells and one electron in the $5S$ 
shell (the number $5$ corresponds to the principal quantum number). Figure~\ref{fig:fs_hfs}a shows the energy levels of a $5S$ and $5P$ orbital.

The atomic level structure is modified by a fine structure splitting of the electronic orbitals into levels with different total electronic angular momentum $\mathbf{J}=\mathbf{L}\cdot\mathbf{S}$. This effect  arises from a spin-orbit interaction between the electron's spin and orbital angular momentum $\hat H_{fs} \propto \mathbf{L}\cdot\mathbf{S}$. Figure~\ref{fig:fs_hfs}b show the $5^2S_{1/2}$, $5^2P_{1/2}$ and $5^2P{3/2}$ electronic configurations that arise from this splitting.  The atomic level structure gets further modified by the nuclear spin $\mathbf{I}$ which interacts with the electron's intrinsic magnetic flux density through the magnetic dipole interaction to give rise to the hyperfine splitting. For $S$ electrons the hyperfine splitting can be described by the Hamiltonian $\hat H_{hfs} = A_{hfs}\mathbf{I}\cdot\mathbf{J}$\footnote{Notice how both the fine and hyperfine structure arise from a spin-orbit coupling interaction, we will discuss a very different type of spin-orbit coupling in future chapter.}. Figure~\ref{fig:fs_hfs}c shows the fine structure getting further split into states of total angular momentum $\mathbf{F}=\mathbf{J}+\mathbf{I}$. $\Rb87$ has a nuclear spin $I=3/2$ and therefore its ground state hyperfine configuration has $F=1$ and $F=2$. 


Something about what we use this transitions for and how we usually ignore all other levels. 


\section{Atom-light interaction}
In the presence of an electric field $\mathbf E$ an atom can become polarized and therefore it's energy levels get shifted through the Stark effect~\cite{stark_beobachtungen_1914}. If the electric field is spatially uniform with respect to the atom's size the effect of the electric field on the atom can be described by the the Hamiltonian   
%
\begin{equation}
\hat{H} = -\mathbf{\hat d}\cdot\mathbf{E},
\label{eq:dipole_ham}	
\end{equation}
%
where $\mathbf{\hat d}=-e\sum_j r_j$ is the atomic dipole operator, $e$ is the electron charge and $\hat r_j$ are the position operators of the atom's electrons relative to the center of mas of the atom. This approximation, known as the dipole approximation, treats the atom as a quantum object and the electric field as a classical object. Here I will only consider the case of oscillating electric fields $\mathbf{E}=E_0\cos(\omega t)\boldsymbol{\epsilon}$ (i.e. plane waves of electromagnetic radiation) which are relevant to our experiments. 

Can break interaction into scalar, vector and tensor part. Interaction can also be resonant or off-resonant.

\note{Why do you only get second order perturbation theory effects? I think it has something to do with unperturbed atomic states being eigenstates of the parity operator}

\subsection{Scalar light shift: Dipole traps and optical lattices}

\subsection{Vector light shift: Raman coupling}

\section{Magnetic interaction}
\subsection{Static magnetic fields}
Also talk about the Paschen–Back effect occurs in a strong external magnetic field. The spin and orbital angular momen- tum precess independently about the magnetic field.

Uniform fields: Zeeman splitting, Breit-Rabi formula
Gradients: Quadrupole potentials and Stern-Gerlach

\subsection{Oscillatory magnetic fields}
RF coupling and microwave coupling

\subsection{Selection rules}

\section{Absorption imaging}
\subsection{Time of flight imaging}
\subsection{Partial transfer absorption imaging: magnetic field stabilization}

\section{Floquet}
How to treat systems when RWA is not valid and how to create new effective (stroboscopic) Hamiltonians.