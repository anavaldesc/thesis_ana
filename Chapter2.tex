% !TEX root = mainthesis.tex
%Chapter 2

\renewcommand{\thechapter}{2}

\chapter{Basic theory of Bose-Einstein condensation}

Bose-Einstein condensation (BEC) is a quantum state of mater in which particles with integer valued spin all tend to occupy or `condense' into the ground state. In dilute gases, condensation occurs when the temperature of the system goes bellow a critical temperature where the bosons become indistinguishable particles and quantum statistics become relevant. 

BECs enable the observation of macroscopic quantum phenomena and there have been a number of fascinating experiments studying the properties of this systems, from measuring interference fringes from a macroscopic wave function to studying collective effects such as the propagation of sound~\cite{ketterle_w._making_1999}, as well as extensive theoretical developments~\cite{dalfovo_theory_1999}. In our experiments however BECs are not the primary object of study, instead they are used as the platform for performing quantum simulations. 

In this Chapter I describe the basic properties of Bose-Einstein condensation in dilute atomic gases. First I will describe the case of an ideal gas and then consider the effects of interactions and a trapping potential. A reader interested in learning about this subject in more depth is advised to read~\cite{Pethick} and~\cite{noauthor_bose-einstein_2003}.

\section{Bose-Einstein condensation of an ideal gas}

At low temperatures and in thermodynamic equilibrium, the mean occupation number of non-interacting identical bosons occupying the state with energy $E$ is given by the Bose distribution
%
\begin{equation}
	n(E_j)=\frac{1}{e^{(E_j-\mu)/k_BT}-1}
	\label{eq:Bose_distribution}	
\end{equation}
%
where $T$ is the temperature, $\mu$ is the chemical potential (the energy cost of adding or removing a particle) and $k_B$ is the Boltzmann constant. In the limit of large temperatures the Bose distribution can be approximated by the Maxwell-Boltzmann distribution
%
\begin{equation}
	n(E_j)\approx e^{-(E_j-\mu)/kT}
\end{equation}
%
which applies to classical, distinguishable particles. The chemical potential is determined by the condition that the total number of particles $N$ is equal to the sum over all states in the distribution $N=\sum_jn(E_j)$ and is therefore a function of $N$ and $T$. Additionally, in order for $n(E_j)$ to be positive definite we must have $\mu\leq E_0$ where $E_0$ is the energy of the ground state. From the Bose distribution we can see that the occupation number of the ground state is unbounded when $\mu\rightarrow0$. As shown in Figure \note{TODO: make figure of Bose distribution} the temperature decreases the occupation number in the ground state increases until all atoms collapse into the lowest energy level and Bose condense. 

\subsection{Critical temperature}
Now I will derive an analytical expression for the critical temperature at which atoms condense. For closely spaced energy levels (compared to $k_B T$) the sum representing the total number of particles can be replaced by the integral
%
\begin{equation}
	N=\int_0^\infty n(E) g(E) dE
	\label{eq:N(E)}
\end{equation}
%
where $g(E)$ is the density of states and $g(E)dE$ corresponds to the number of available states with energy between $E$ and $E+dE$. For a free particle in three dimensions the density of states is
%
\begin{equation}
	g(E)=\frac{V m^{3/2}}{\sqrt{2}\pi^2\hbar^2}E^{1/2},
	\label{eq:free_particle_dos}
\end{equation}
%
and in general the density of states can be expressed as a power of energy $g(E)=C_\alpha E^{\alpha-1}$. 

The integral in Equation~\ref{eq:N(E)} is not analytically solvable, however we can make the simplifying assumption $\mu=0$. The critical temperature $T_c$ is determined by the condition that all particles are in the excited states
%
\begin{align}
	N&=N_{\rm{ex}}(T_c, \mu=0) \nonumber \\
	&=\int_0^{\infty}\frac{g(E)dE}{e^{E/k_BT_c}-1} \nonumber \\
	&= C_\alpha(k_BT_c)^\alpha\int_0^\infty\frac{x^{\alpha-1}}{e^x-1} \nonumber \\
	&= c_\alpha (k_BT_c)^\alpha\Gamma(\alpha)\zeta(\alpha)
	\label{eq:finding_Tc}
\end{align}
%
where I made the substitution $x=E/k_BT_c$, $\Gamma(\alpha)=\int_0^\infty x^{\alpha-1}e^{-x}dx$ is the Gamma function and $\zeta(\alpha)=\sum_{n=1}^\infty n^{-\alpha}$ is the Riemann zeta function. From Equation~\ref{eq:finding_Tc} we find that the critical temperature for Bose-Einstein condensation is
%
\begin{equation}
	k_BT_c=\left(\frac{N}{C_\alpha\Gamma(\alpha)\zeta(\alpha)}\right)^{1/\alpha}.
\end{equation}

As mentioned earlier, Bose-Einstein condensation can be understood in terms of the de Broglie waves associated to particles. The thermal de Broglie wavelength is defined as
%
\begin{equation}
	\lambda_{\rm{th}}=\left(\frac{2\pi\hbar^2}{mk_BT}\right)^{1/2}
\end{equation}
%
and it characterized the spatial extension of the wave packet an individual particle at temperature $T$. Condensation occurs when $\lambda_{\rm{th}}$ becomes comparable with the inter-particle separation $n^{-1/3}$, where $n=N/V$. Using the density of states for a free particle in 3D (Equation~\ref{eq:free_particle_dos})in combination with the expression for the critical temperature (Equation~\ref{eq:finding_Tc}) we find that indeed when $T=T_c$
%
\begin{equation}
	n\lambda_{\rm {th}}^3=\zeta\left(\frac{3}{2}\right)\approx 2.612
\end{equation}
%
both quantities are comparable. The quantity $n\lambda_{\rm{th}}^3$ is known as the phase space density which describes the number of particles contained in a box with volume $\lambda_{\rm{th}}^3$. In order to experimentally produce BECs, a combination of laser and evaporative cooling techniques are deployed such that we can increase the density while minimizing the temperature and therefore maximize the phase space density. The densities for BECs of Alkali atoms typically range in of order $10^{13}$ to $10^{15}$ atoms/cm$^{-3}$.

\subsection{Condensate fraction}

Now we look at the fraction of particles occupying the ground state at temperatures below $T_c$. The total number of particles is given by $N=N_0+N_{\rm{ex}}$. The number of particles in the excited state will be given by the integral in Equation~\ref{eq:N(E)}. For $g(E)=C_\alpha E^{\alpha-1}$and  $\alpha>0$ the integral converges and we get
%
\begin{align}
	N_{\rm{ex}}&=c_\alpha (k_BT)^\alpha\Gamma(\alpha)\zeta(\alpha) \nonumber \\
	&=N\left(\frac{T}{T_c}\right)^\alpha,
\end{align}
%
where I used the expression for the total number of particles at $T_c$ from Equation~\ref{eq:finding_Tc}. The number of particles in the ground state is therefore
%
\begin{align}
	N_0&=N-N_{\rm{ex}} \nonumber \\
	&= N\left[1-\left(\frac{T}{T_c}\right)^\alpha\right]
\end{align}

\subsection{Bose gas in a harmonic trapping potential}

I consider the particular case of particles confined in a three dimensional harmonic potential
%
\begin{equation}
U(\x)=\frac{m}{2}\left(\omega_x^2x^2+\omega_y^2y^2+\omega_z^2z^2\right)
\end{equation}
%
as it is the most relevant to our experiments that are performed in harmonic traps. The density of states is given by 

\begin{equation}
	g(E)=\frac{E^2}{2\hbar^2\omega_x\omega_y\omega_z},
\end{equation}
%
which corresponds to $\alpha=3$ and $C_3=(2\hbar^3\omega_x\omega_y\omega_z)^{-1}$. Using Equation~\ref{eq:finding_Tc}, the transition temperature is
%
\begin{equation}
 	k_B T_c=\frac{\hbar \bar{\omega}N^{1/3}}{\zeta(3)^{1/3}}\approx0.94\hbar\bar{\omega}N^{1/3}
 \end{equation} 
%
where $\bar{\omega}=(\omega_x\omega_y\omega_z)^{1/3}$ is the geometric mean of the oscillation frequencies. Similarly we find that the condensed fraction is
%
\begin{equation}
	N_0=N\left[1-\left(\frac{T}{T_c}\right)^3\right]
\end{equation}

Condensates in harmonic traps have some striking features that will be further explored in more detail in the following sections. The confining potential makes the BECs both finite sized and inhomogeneous which means that the BEC can be observed both in momentum space and in coordinate space. Another consequence of the inhomogeneity of these systems is the role of two-body interactions, which gets enhanced and leads to noticeable effects in measurable quantities (see~\cite{dalfovo_theory_1999,castin_bose-einstein_1996}).

\section{Bose-Einstein condensation with atomic interactions}

Even though atomic BECs are made from very dilute gases, the system is far from being an ideal gas and interactions need to be taken into account for a complete treatment of the system. 

Hartree-Fock approximation means write the many body wave function as a single component
\subsection{GPE equation}

The natural starting point for studying the behavior of
these systems is the theory of weakly interacting bosons
which, for inhomogeneous systems, takes the form of
the Gross-Pitaevskii theory. This is a mean-field approach for the order parameter associated with the condensate. It provides closed and relatively simple equations for describing the relevant phenomena associated
with BEC. In particular, it reproduces typical properties
exhibited by superfluid systems, like the propagation of
collective excitations and the interference effects originating from the phase of the order parameter. The
theory is well suited to describing most of the effects of
two-body interactions in these dilute gases at zero temperature and can be naturally generalized to also explore thermal effects.
\subsection{Thomas-Fermi approximation}
\section{Density and momentum distributions}
Why the BEC has the shape of an inverted parabola. 
\subsection{Expansion of atomic cloud in 3D harmonic potential}
How one can infer atomic densities and temperature (from thermal atoms) from time of flight images.
\section{Density profiles}


For a given trapping potential $U(\r)$, the density distribution of a thermal ensamble is
\begin{equation}
	n(\r) = n_0 e^{-\frac{U(\r)}{k_BT}}.
\end{equation}
%
The temperature $T$ can be derived from the density distribution. For a 3D harmonic trap
%
\begin{equation}
	n(\r)=n_0e^{-(\frac{x^2}{2\sigma_x^2}+\frac{y^2}{2\sigma_y^2}+\frac{z^2}{2\sigma_z^2})},
\end{equation}
%
where $\sigma_i=\omega_i^{-1}\sqrt{k_BT/m}$. Using the equipartition theorem we find that the spatial extension of the cloud and the temperature are related by 
%
\begin{equation}
	T=\frac{m}{k_B}\sigma_i^2\omega_i^2
\end{equation}

\note{TODO: what is my chemical potential?}
%Probably I don't want to talk about any of this stuff...
% \section{Laser cooling and trapping techniques}

% \subsection{Zeeman slower}
% \subsection{Magneto-optical trap}
% \subsection{Optical molasses and sub dopler cooling}
% \subsection{RF induced evaporation}
% \subsection{Dipole trap}
% \subsection{Evaporation in the dipole trap}




