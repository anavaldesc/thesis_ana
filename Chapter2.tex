% !TEX root = mainthesis.tex
%Chapter 2

\renewcommand{\thechapter}{2}

\chapter{Basic theory of Bose-Einstein condensation}

Bose-Einstein condensation (BEC) is a quantum state of mater in which particles with integer valued spin all tend to occupy or `condense' into the ground state. In dilute gases, condensation occurs when the temperature of the system goes bellow a critical temperature where the bosons become indistinguishable particles and quantum statistics become relevant. 

BECs enable the observation of macroscopic quantum phenomena and there have been a number of fascinating experiments studying the properties of this systems, from measuring interference fringes from a macroscopic wave function to studying collective effects such as the propagation of sound~\cite{ketterle_w._making_1999}, as well as extensive theoretical developments~\cite{dalfovo_theory_1999}. In our experiments however BECs are not the primary object of study, instead they are used as the platform for performing quantum simulations. 

In this Chapter I describe the basic properties of Bose-Einstein condensation in dilute atomic gases. First I will describe the case of an ideal gas and then consider the effects of interactions and a trapping potential. A reader interested in learning about this subject in more depth is advised to read~\cite{Pethick} and~\cite{noauthor_bose-einstein_2003}.

\section{The Bose-Einstein distribution}

At low temperatures and in thermodynamic equilibrium, the mean occupation number of non-interacting identical bosons occupying the state with energy $E$ is given by the Bose distribution
%
\begin{equation}
	n(E_j)=\frac{1}{e^{(E_j-\mu)/k_BT}-1}
	\label{eq:Bose_distribution}	
\end{equation}
%
where $T$ is the temperature, $\mu$ is the chemical potential (the energy cost of adding or removing a particle) and $k_B$ is the Boltzmann constant. The chemical potential is constrained to take only positive values in order for Equation~\ref{eq:Bose_distribution} to take positive values and it is a function of the total number of particles $N$ and $T$.

Bose-Einstein condensation as a result of the Bose statistics. When 



In the limit of large temperatures the Bose distribution can be approximated by the Maxwell-Boltzmann distribution
%
\begin{equation}
	n(E_j)\approx e^{-(E_j-\mu)/kT}
\end{equation}
%
which applies to classical, distinguishable particles. 

Condensation occurs when the chemical potential is equal to the energy of the ground state

A naive approach: $\lambda_T=\left(\frac{2\pi\hbar^2}{mk_BT}\right)^{1/2}$ comparable to the mean interparticle spacinc which is of order $n^{-1/3}$ (in Alkalis $10^{13}$ to $10^{15}$ cm$^{-3}$)

For closely spaced energy levels (compared to $k_B T$) the total number of particles is given by the integral
%
\begin{equation}
	N=\int_0^\infty n(E) g(E) dE
\end{equation}
%
where $g(E)$ is the density of states, and $g(E)dE$ corresponds to the number of available states with energy between $E$ and $E+dE$. For a free particle in three dimensions the density of states is
%
\begin{equation}
	g(E)=\frac{V m^{3/2}}{\sqrt{2}\pi^2\hbar^2}E^{1/2}
\end{equation}
%
and for harmonic potential
%
\begin{equation}
	g(E)=\frac{E^2}{2\hbar^2\omega_x\omega_y\omega_z}
\end{equation}


\section{BEC transition and critical temperature}

Macroscopic occupation of the ground state
\section{BEC in a harmonic potential}
\section{BEC with interactions}
\subsection{GPE equation}
\subsection{Thomas-Fermi approximation}
Why the BEC has the shape of an inverted parabola. 
\subsection{Expansion of atomic cloud in 3D harmonic potential}
How one can infer atomic densities and temperature (from thermal atoms) from time of flight images.
\section{Density profiles}


For a given trapping potential $U(\r)$, the density distribution of a thermal ensamble is
\begin{equation}
	n(\r) = n_0 e^{-\frac{U(\r)}{k_BT}}.
\end{equation}
%
The temperature $T$ can be derived from the density distribution. For a 3D harmonic trap
%
\begin{equation}
	n(\r)=n_0e^{-(\frac{x^2}{2\sigma_x^2}+\frac{y^2}{2\sigma_y^2}+\frac{z^2}{2\sigma_z^2})},
\end{equation}
%
where $\sigma_i=\omega_i^{-1}\sqrt{k_BT/m}$. Using the equipartition theorem we find that the spatial extension of the cloud and the temperature are related by 
%
\begin{equation}
	T=\frac{m}{k_B}\sigma_i^2\omega_i^2
\end{equation}

\note{TODO: what is my chemical potential?}
%Probably I don't want to talk about any of this stuff...
% \section{Laser cooling and trapping techniques}

% \subsection{Zeeman slower}
% \subsection{Magneto-optical trap}
% \subsection{Optical molasses and sub dopler cooling}
% \subsection{RF induced evaporation}
% \subsection{Dipole trap}
% \subsection{Evaporation in the dipole trap}




