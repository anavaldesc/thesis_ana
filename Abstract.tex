% !TEX root = mainthesis.tex
%Abstract Page 

\hbox{\ }

\renewcommand{\baselinestretch}{1}
\small \normalsize

\begin{center}
\large{{ABSTRACT}} 

\vspace{3em} 

\end{center}
\hspace{-.15in}
\begin{tabular}{ll}
Title of dissertation:   
&				      {\large  TOPOLOGICAL DISPERSION RELATIONS IN } \\
&				      {\large  SPIN-ORBIT COUPLED BOSE GASES} \\
\ \\
&                     {\large  Ana Valdés-Curiel,} \\
&					  {\large  Doctor of Philosophy, 2019} \\
\ \\
Dissertation directed by: & {\large  Professor Ian Spielman} \\
&  							{\small	 Joint Quantum Institute,} \\
&  							{\small	 National Institute of Standards and Technology and} \\
&  							{\small	 University of Maryland College Park} \\
\end{tabular}

\vspace{3em}

\renewcommand{\baselinestretch}{2}
\large \normalsize

Quantum degenerate gases have proven to be an ideal platform for the  simulation of complex systems. Due to their high level of control it is possible to readily design and implement systems with effective Hamiltonians in the laboratory. This thesis focuses on the development of new tools for the characterization and control of engineered quantum systems, and applies them to create and characterize a topological system with Rashba-type spin-orbit coupling. 

The underlying properties of such engineered systems depend on their single particle energies and it is therefore important to characterize them. I describe a Fourier transform spectroscopy technique to probe the single particle spectrum of a quantum system and apply it to measure the dispersion relation of a spin-1 spin-orbit coupled Bose-Einstein condensate (BEC). Fourier transform spectroscopy relies only on measuring the unitary evolution under a Hamiltonian of interest and can be applied generically to any system with long enough coherent evolution.

Decoherence of quantum systems due to uncontrolled fluctuations of the environment presents fundamental obstacles in quantum science. I describe an implementation of continuous dynamical decoupling (CDD) in a spin-1 BEC. We applied a strong radio-frequency (RF) magnetic field to the ground state hyperfine manifold of Rubidium-87 atoms, generating a dynamically protected dressed system that was first-order insensitive to changes in magnetic field. The CDD states constitute effective clock states and we observed a reduction in sensitivity to magnetic field of up to four orders of magnitude. I show that the CDD states can be coupled in a fully connected way unlike bare atomic states which are constrained by selection rules. 

Finally, I describe the quantum engineering of Rashba-type SOC using Raman coupled CDD states. Our engineered system had non-trivial topology but without an underlying crystalline structure that yields integer Chern numbers in conventional materials. We validated our procedure using Fourier transform spectroscopy to measure the full dispersion relation that contained only a single Dirac point. We measured the quantum geometry underlying the dispersion relation and obtained the topological index using matter-wave interferometry. In contrast to crystalline materials, where topological indices take on integer values, our continuum system reveals an unconventional half-integer Chern number. 

% In particular,  . 